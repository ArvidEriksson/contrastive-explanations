% DO NOT EDIT - automatically generated from metadata.yaml

\def \codeURL{https://github.com/ArvidEriksson/contrastive-explanations}
\def \codeDOI{}
\def \codeSWH{}
\def \dataURL{}
\def \dataDOI{}
\def \editorNAME{}
\def \editorORCID{}
\def \reviewerINAME{}
\def \reviewerIORCID{}
\def \reviewerIINAME{}
\def \reviewerIIORCID{}
\def \dateRECEIVED{01 November 2018}
\def \dateACCEPTED{}
\def \datePUBLISHED{}
\def \articleTITLE{[Re] Reproducibility review of “Why Not Other Classes?”: Towards Class-Contrastive Back-Propagation Explanations}
\def \articleTYPE{Replication}
\def \articleDOMAIN{Computer Vision}
\def \articleBIBLIOGRAPHY{bibliography.bib}
\def \articleYEAR{2025}
\def \reviewURL{}
\def \articleABSTRACT{“Why Not Other Classes?”: Towards Class-Contrastive Back-Propagation Explanations (Wang \& Wang, 2022) provides a method for contrastively explaining why a certain class in a neural network image classifier is chosen above others. This method consists of using back- propagation-based explanation methods from after the softmax layer rather than before. Our work consists of reproducing the work in the original paper. We also provide extensions to the paper by evaluating the method on XGradCAM, FullGrad, and Vision Transformers to evaluate its generalization capabilities. The reproductions show similar results as the original paper, with the only difference being the visualization of heatmaps which could not be reproduced to look similar. The generalization seems to be generally good, with imple- mentations working for Vision Transformers and alternative back-propagation methods. We also show that the original paper suffers from issues such as a lack of detail in the method and an erroneous equation which makes reproducibility difficult. To remedy this we provide an open-source repository containing all code used for this project.}
\def \replicationCITE{Wang, Yipei, and Xiaoqian Wang. "Why Not Other Classes?" Towards Class-Contrastive Back-Propagation Explanations. Advances in Neural Information Processing Systems, 6 Dec. 2022.}
\def \replicationBIB{wang2022why}
\def \replicationURL{https://proceedings.neurips.cc/paper_files/paper/2022/file/3b7a66b2d1258e892c89f485b8f896e0-Paper-Conference.pdf}
\def \replicationDOI{}
\def \contactNAME{Arvid Eriksson}
\def \contactEMAIL{arveri@kth.se}
\def \articleKEYWORDS{rescience c, rescience x, python, explainability, computer vision}
\def \journalNAME{ReScience C}
\def \journalVOLUME{4}
\def \journalISSUE{1}
\def \articleNUMBER{}
\def \articleDOI{}
\def \authorsFULL{Arvid Eriksson, Anton Israelsson and Mattias Kallhauge}
\def \authorsABBRV{A. Eriksson, A. Israelsson and M. Kallhauge}
\def \authorsSHORT{Eriksson, Israelsson and Kallhauge}
\title{\articleTITLE}
\date{}
\author[1,\orcid{0009-0006-9434-0277}]{Arvid Eriksson}
\author[1]{Anton Israelsson}
\author[1]{Mattias Kallhauge}
\affil[1]{KTH Royal Institute of Technology, Department of Computer Science, Stockholm, Sweden}
